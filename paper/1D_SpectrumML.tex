%% Copyright 2019-2021 Elsevier Ltd
%% 
%% This file is part of the 'CAS Bundle'.
%% --------------------------------------
%% 
%% It may be distributed under the conditions of the LaTeX Project Public
%% License, either version 1.2 of this license or (at your option) any
%% later version.  The latest version of this license is in
%%    http://www.latex-project.org/lppl.txt
%% and version 1.2 or later is part of all distributions of LaTeX
%% version 1999/12/01 or later.
%% 
%% The list of all files belonging to the 'CAS Bundle' is
%% given in the file `manifest.txt'.
%% 
%% Template article for cas-sc documentclass for 
%% single column output.

\documentclass[a4paper,fleqn]{cas-sc}

% If the frontmatter runs over more than one page
% use the longmktitle option.

%\documentclass[a4paper,fleqn,longmktitle]{cas-sc}

%\usepackage[numbers]{natbib}
%\usepackage[authoryear]{natbib}
\usepackage[authoryear,longnamesfirst]{natbib}

%%%Author macros
\def\tsc#1{\csdef{#1}{\textsc{\lowercase{#1}}\xspace}}
\tsc{WGM}
\tsc{QE}
%%%

% Uncomment and use as if needed
%\newtheorem{theorem}{Theorem}
%\newtheorem{lemma}[theorem]{Lemma}
%\newdefinition{rmk}{Remark}
%\newproof{pf}{Proof}
%\newproof{pot}{Proof of Theorem \ref{thm}}

\begin{document}
\let\WriteBookmarks\relax
\def\floatpagepagefraction{1}
\def\textpagefraction{.001}

% Short title
\shorttitle{Machine Learning wave prediction with Spectra}    

% Short author
\shortauthors{LPeach, NCartwright, GVieraDeSilva}  

% Main title of the paper
\title [mode = title]{Machine Learning wave prediction using spectra}  

% Title footnote mark
% eg: \tnotemark[1]
\tnotemark[<tnote number>] 

% Title footnote 1.
% eg: \tnotetext[1]{Title footnote text}
\tnotetext[<tnote number>]{<tnote text>} 

% First author
%
% Options: Use if required
% eg: \author[1,3]{Author Name}[type=editor,
%       style=chinese,
%       auid=000,
%       bioid=1,
%       prefix=Sir,
%       orcid=0000-0000-0000-0000,
%       facebook=<facebook id>,
%       twitter=<twitter id>,
%       linkedin=<linkedin id>,
%       gplus=<gplus id>]

\author[<aff no>]{Leo Peach}

% Corresponding author indication
\cormark[<corr mark no>]

% Footnote of the first author
\fnmark[<footnote mark no>]

% Email id of the first author
\ead{leo.peach@griffithuni.edu.au}

% URL of the first author
\ead[url]{<URL>}

% Credit authorship
% eg: \credit{Conceptualization of this study, Methodology, Software}
\credit{Methodology, Scientific writing, literature review, software}

% Address/affiliation
\affiliation[<aff no>]{organization={Griffith University},
            addressline={Southport}, 
            city={Gold Coast},
%          citysep={}, % Uncomment if no comma needed between city and postcode
            postcode={0000}, 
            state={Queensland},
            country={Australia}}

\author[<aff no>]{Leo Peach}

% Footnote of the second author
\fnmark[2]

% Email id of the second author
\ead{}

% URL of the second author
\ead[url]{}

% Credit authorship
\credit{Leo Peach}

% Address/affiliation
\affiliation[<aff no>]{organization={Griffith University},
            addressline={Southport}, 
            city={Gold Coast},
%          citysep={}, % Uncomment if no comma needed between city and postcode
            postcode={00000}, 
            state={Queensland},
            country={Australia}}

% Corresponding author text
\cortext[1]{Leo Peach}

% Footnote text
\fntext[1]{}

% For a title note without a number/mark
%\nonumnote{}
% Here goes the abstract
\begin{abstract}
Here we describe a method for improving predicting using a 1D wave spectrum.
\end{abstract}

% Use if graphical abstract is present
%\begin{graphicalabstract}
%\includegraphics{}
%\end{graphicalabstract}

% Research highlights
\begin{highlights}
\item Machine Learning
\item Wave Prediction
\item Spectra
\end{highlights}

% Keywords
% Each keyword is seperated by \sep
\begin{keywords}
Machine Learning \sep Wave Prediction \sep Spectra
\end{keywords}

\maketitle

% Main text
\section{Introduction}\label{Intro}
The forecasting of wave heights is important for a range of maritime and coastal applications, from port operations through to beach management and surfing. One of the challenges is the ability to downscale predictions from oceanographic wave models to the coast. This is required as additional physical processes occur as the waves approach local features, such as depth induced wave breaking, refraction and defraction. Typically this involves the development of computationally expensive physics based models, which also require good quality input data, including bathymetry to perform well.
Data-driven approaches have increasingly been applied in order to help overcome the computational cost and allow the rapid downscaling of wave forecasts and hindcasts. Most have utilised wave parameters <<<<REFS HERE>>>> , but this can have it's challenges as wave parameters are summaries of the wave spectrum (which is a more full description of wave conditions in the form of energy spread over a range of frequencies)<<<REFS>>>. 
Machine learning techniques are being increasingly applied in this space


\section{Method}\label{method}


\section{Discussion}\label{disc}


\section{Conclusion}\label{con}


% Numbered list
% Use the style of numbering in square brackets.
% If nothing is used, default style will be taken.
%\begin{enumerate}[a)]
%\item 
%\item 
%\item 
%\end{enumerate}  

% Unnumbered list
%\begin{itemize}
%\item 
%\item 
%\item 
%\end{itemize}  

% Description list
%\begin{description}
%\item[]
%\item[] 
%\item[] 
%\end{description}  

% Figure
%\begin{figure}[<options>]
%	\centering
%		\includegraphics[<options>]{}
%	  \caption{}\label{fig1}
%\end{figure}


%\begin{table}[<options>]
%\caption{}\label{tbl1}
%\begin{tabular*}{\tblwidth}{@{}LL@{}}
%\toprule
%  &  \\ % Table header row
%\midrule
% & \\
% & \\
% & \\
% & \\
%\bottomrule
%\end{tabular*}
%\end{table}

% Uncomment and use as the case may be
%\begin{theorem} 
%\end{theorem}

% Uncomment and use as the case may be
%\begin{lemma} 
%\end{lemma}

%% The Appendices part is started with the command \appendix;
%% appendix sections are then done as normal sections
%% \appendix

\section{}\label{}

% To print the credit authorship contribution details
%\printcredits

%% Loading bibliography style file
%\bibliographystyle{model1-num-names}
\bibliographystyle{cas-model2-names}

% Loading bibliography database
\bibliography{}

% Biography
\bio{}
Leo is a PhD candidate at the Griffith Univeristy School of Engineering and Built Environment, and the Marine and Coastal Research Centre
\endbio

%\bio{}
%% Here goes the biography details.
%\endbio

\end{document}

