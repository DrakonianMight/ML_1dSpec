\documentclass[review]{elsarticle}
\usepackage{tikz}
% Define custom colors to match Word-like colors
\definecolor{seaBlue}{RGB}{91, 155, 213}    % Similar to Word's blue
\definecolor{swellCoral}{RGB}{237, 125, 49} % Similar to Word's orange/coral
\definecolor{outputGray}{RGB}{191, 191, 191} % Similar to Word's light gray
\begin{document}

\begin{figure}[!htbh]
\usetikzlibrary{shapes.geometric, arrows}
\tikzstyle{output} = [rectangle, rounded corners, minimum width=3cm, minimum height=1cm,text centered, draw=black, fill=outputGray, text width=3cm, inner sep=3pt]
\tikzstyle{io} = [rectangle, rounded corners, minimum width=4.5cm, minimum height=1cm, text centered, draw=black, fill=seaBlue, text width=4.5cm, inner sep=3pt]
\tikzstyle{process} = [rectangle, minimum width=3cm, minimum height=1cm, text centered, draw=black, fill=green!30, text width = 3cm, inner sep=3pt]
\tikzstyle{decision} = [diamond, minimum width=3cm, minimum height=1cm, text centered, draw=black, fill=green!30]
\tikzstyle{arrow} = [thick,->,>=stealth]
\begin{tikzpicture}[node distance=1.4cm]
%% Our nodes
\node (Hs) [io, xshift=-10cm, yshift=-10] {H\textsubscript{s} \textsubscript{O} \\ (t = t, t -1dt, ... , t -12dt)};
\node (Tp) [io, below of=Hs] {T\textsubscript{p} \textsubscript{O} \\ (t = t, t -1dt, ... , t -12dt)};
\node (Tz) [io, below of=Tp] {T\textsubscript{z} \textsubscript{O} \\ (t = t, t -1dt, ... , t -12dt)};
\node (Dp_cos) [io, below of=Tz] {cos(D\textsubscript{p} \textsubscript{O}) \\ (t = t, t -1dt, ... , t -12dt)};
\node (Dp_sin) [io, below of=Dp_cos] {sin(D\textsubscript{p} \textsubscript{O}) \\ (t = t, t -1dt, ... , t -12dt)};
\node (Hs_n) [output, right of=Hs, xshift=5cm, yshift=-2.5cm] {H\textsubscript{s} NS(t)}; 


%% our arrows
\draw [arrow] (Hs) -- (Hs_n);
\draw [arrow] (Tp) -- (Hs_n);
\draw [arrow] (Tz) -- (Hs_n);
\draw [arrow] (Dp_cos) -- (Hs_n);
\draw [arrow] (Dp_sin) -- (Hs_n);
\end{tikzpicture}
\caption{Diagram showing the architecture of the training and validation datasets, features are on the left and the label or predictor (in this case H\textsubscript{s}) is on the right. Where \textsubscript{O} represents Offshore, NS Nearshore, t represents the predicted timestep and dt represents a timestep (in this case 1 hour). Note that the label or predictor for direction will be two parameters cos(D\textsubscript{p}) and sin(D\textsubscript{p}).}
\label{fig:NNTrainingDiagram}
\end{figure}


\end{document}
